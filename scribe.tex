%
% This is the LaTeX template file for lecture notes for EE 382C/EE 361C.
%
% To familiarize yourself with this template, the body contains
% some examples of its use.  Look them over.  Then you can
% run LaTeX on this file.  After you have LaTeXed this file then
% you can look over the result either by printing it out with
% dvips or using xdvi.
%
% This template is based on the template for Prof. Sinclair's CS 270.

\documentclass[twoside]{article}
\usepackage{graphics}
\usepackage{enumitem}
\setlength{\oddsidemargin}{0.25 in}
\setlength{\evensidemargin}{-0.25 in}
\setlength{\topmargin}{-0.6 in}
\setlength{\textwidth}{6.5 in}
\setlength{\textheight}{8.5 in}
\setlength{\headsep}{0.75 in}
\setlength{\parindent}{0 in}
\setlength{\parskip}{0.1 in}

%
% The following commands set up the lecnum (lecture number)
% counter and make various numbering schemes work relative
% to the lecture number.
%
\newcounter{lecnum}
\renewcommand{\thepage}{\thelecnum-\arabic{page}}
\renewcommand{\thesection}{\arabic{section}}
\renewcommand{\theequation}{\thelecnum.\arabic{equation}}
\renewcommand{\thefigure}{\thelecnum.\arabic{figure}}
\renewcommand{\thetable}{\thelecnum.\arabic{table}}

%
% The following macro is used to generate the header.
%
\newcommand{\lecture}[4]{
   \pagestyle{myheadings}
   \thispagestyle{plain}
   \newpage
   \setcounter{lecnum}{#1}
   \setcounter{page}{1}
   \noindent
   \begin{center}
   \framebox{
      \vbox{\vspace{2mm}
    \hbox to 6.28in { {\bf EE 382C/361C: Multicore Computing
                        \hfill Fall 2016} }
       \vspace{4mm}
       \hbox to 6.28in { {\Large \hfill Questions for Concurrent Queues, Stacks and Linked Lists \hfill} }
       \vspace{2mm}
       \hbox to 6.28in { {\it Lecturer: #3 \hfill Scribe: #4} }
      \vspace{2mm}}
   }
   \end{center}
   \markboth{Lecture #1: #2}{Lecture #1: #2}
   %{\bf Disclaimer}: {\it These notes have not been subjected to the
   %usual scrutiny reserved for formal publications.  They may be distributed
   %outside this class only with the permission of the Instructor.}
   \vspace*{4mm}
}

%
% Convention for citations is authors' initials followed by the year.
% For example, to cite a paper by Leighton and Maggs you would type
% \cite{LM89}, and to cite a paper by Strassen you would type \cite{S69}.
% (To avoid bibliography problems, for now we redefine the \cite command.)
% Also commands that create a suitable format for the reference list.
\renewcommand{\cite}[1]{[#1]}
\def\beginrefs{\begin{list}%
        {[\arabic{equation}]}{\usecounter{equation}
         \setlength{\leftmargin}{2.0truecm}\setlength{\labelsep}{0.4truecm}%
         \setlength{\labelwidth}{1.6truecm}}}
\def\endrefs{\end{list}}
\def\bibentry#1{\item[\hbox{[#1]}]}

%Use this command for a figure; it puts a figure in wherever you want it.
%usage: \fig{NUMBER}{SPACE-IN-INCHES}{CAPTION}
\newcommand{\fig}[3]{
      \vspace{#2}
      \begin{center}
      Figure \thelecnum.#1:~#3
      \end{center}
  }
% Use these for theorems, lemmas, proofs, etc.
\newtheorem{theorem}{Theorem}[lecnum]
\newtheorem{lemma}[theorem]{Lemma}
\newtheorem{proposition}[theorem]{Proposition}
\newtheorem{claim}[theorem]{Claim}
\newtheorem{corollary}[theorem]{Corollary}
\newtheorem{definition}[theorem]{Definition}
\newenvironment{proof}{{\bf Proof:}}{\hfill\rule{2mm}{2mm}}

% **** IF YOU WANT TO DEFINE ADDITIONAL MACROS FOR YOURSELF, PUT THEM HERE:
\usepackage{graphicx}
\graphicspath{ {images/} }

\begin{document}
%FILL IN THE RIGHT INFO.
%\lecture{**LECTURE-NUMBER**}{**DATE**}{**LECTURER**}{**SCRIBE**}
\lecture{}{}{Vijay Garg}{Weihao Xu}
%\footnotetext{These notes are partially based on those of Nigel Mansell.}

% **** YOUR NOTES GO HERE:

% Some general latex examples and examples making use of the
% macros follow.  
%**** IN GENERAL, BE BRIEF. LONG SCRIBE NOTES, NO MATTER HOW WELL WRITTEN,
%**** ARE NEVER READ BY ANYBODY.
\section{Questions}
1. To implement lock-free concurrent queue in java, why do we need to use class AtomicStampedReference instead of AtomicReference?

2. Fine-grained synchronization and coarse-grained synchronization, which one costs more memory? Which one is more parallel?
\begin{enumerate} [label=(\Alph*)]
\item fine-grained synchronization, fine-grained synchronization

\item coarse-grained synchronization, coarse-grained synchronization
\item fine-grained synchronization, coarse-grained synchronization
\item coarse-grained synchronization, fine-grained synchronization
\end{enumerate}

3. Suppose we are using AtomicReference class to implement a non-blocking synchronization linked list. The linked list is head $\rightarrow$ a $\rightarrow$ b $\rightarrow$ c. What will happen if two threads A and B try to remove node a and b respectively using CAS operation at the same time?

\begin{enumerate} [label=(\Alph*)]
\item both a and b will be removed
\item neither a nor b will be removed
\item a will be removed, b will not be removed
\item a will not be removed, b will be removed
\end{enumerate}

4. How to solve the problem revealed in Q3?

5. What is the assumption when we adopt optimistic synchronization for concurrent linked list? 

6. In Michael and Scott’s Lock Free Queue Algorithm, what will thread A do as next step in the following two scenarios when it is executing enque(a) operation?
\begin{enumerate} [label=(\alph*)]
\item tail.next.get() == b != null
\item tail.next.get() == null
\end{enumerate}

7. For lock-based concurrent linked list, at least how many times of traverse does it need to complete an add() function when using Optimistic Synchronization and lazy synchronization, respectively?

\section{Answers}

1. To avoid ABA problem, we need to use a integer stamp that is atomically updated whenever a node is updated.

2. A

3. C

4. Use AtomicMarkableReference instead of AtomicReference. Atomically set the mark while removing a node. 

5. In optimistic synchronization, if a synchronization conflict causes the wrong nodes to be locked, then the thread need to release the locks and start over. Therefore, when adopting optimistic synchronization, we suppose this kind of conflict is rare, otherwise the performance would be poor.

6. 
\begin{enumerate} [label=(\alph*)]
\item The ‘next’ field of tail points to node b, indicating that node b is in the process of enque(b) but not finished. Therefore, thread A need to help swing the tail to b and start over.
\item The ‘next’ field of tail points to null, indicating that no other enque() action happens before A. Then thread A need to try to point the next field of tail to node a and then swing tail to a.
\end{enumerate}

7. 
\begin{enumerate} [label=(\alph*)]
\item The add() operation of optimistic synchronization needs at least two traversals, one for navigating and locking the pred and curr nodes, and the other for validating these nodes.
\item The add() operation of lazy synchronization needs at least one traversal.
\end{enumerate}

\section*{References}
\beginrefs
\bibentry{1}{\sc V. K. Garg}, Introduction to Multicore Computing
\endrefs
\beginrefs
\bibentry{2}{\sc Herlihy, Maurice, and Nir Shavit}, The art of multiprocessor programming." PODC. Vol. 6. 2006.
\endrefs

\end{document}